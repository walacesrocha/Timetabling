\documentclass[11pt]{article}

\usepackage{sbpo}
\usepackage{graphicx}
\usepackage{url}

\usepackage[T1]{fontenc}
\usepackage[brazil]{babel}   
\usepackage[utf8]{inputenc}  

\sloppy

\title{Algoritmo Grasp para o Problema de Tabela-horário de Universidades}

% Uncomment next line to hide authors, affiliations and contacts
%\hideauthors

\author{\sbpoauthor{Walace S. Rocha}
                   {Universidade Federal do Espírito Santo}
                   {Vitóra -- ES -- Brasil}
                   {\email{walacesrocha@yahoo.com.br}}
        \sbpoauthor{Maria C. S. Boeres}
                   {Departamento de Informática -- Universidade Federal do Espírito Santo}
                   {Vitória -- ES -- Brasil}
                   {\email{boeres@inf.ufes.br}}
        \sbpoauthor{Maria C. Rangel}
                   {Departamento de Informática -- Universidade Federal do Espírito Santo}
                   {Vitória -- ES -- Brasil}
                   {\email{crangel@inf.ufes.br}}}

\begin{document} 

\maketitle

\palavrachave{Tabela-horário}
\palavrachave{Meta-heurística}
\palavrachave{Grasp}
\areaprincipal{Área de classificação principal (escolher uma na janela de áreas)}

\begin{resumo} 
  Este modelo resume as normas de formatação para os trabalhos completos a serem publicados nos Anais do XLI SBPO. O Resumo deve ter no máximo 150 palavras. Título, nomes dos autores, sua filiação e endereços, resumo e palavras-chave devem repetir fielmente o que foi informado quando o artigo foi cadastrado através do sistema JEMS. No primeiro upload, por ocasião da submissão do trabalho, exclua desta primeira página os nomes, instituições e endereços dos autores.
\end{resumo}

\keyword{Timetabling}
\keyword{Metaheuristic}
\keyword{Grasp}
\mainarea{Main area (choose between those in the areas window)}

\begin{abstract}
  This model summarizes the formatting rules for full papers accepted for inclusion ion the Proceedings of the XLI SBPO. The Abstract must have no more than 150 words Title, author names, affiliations, addresses, abstract and keywords must precisely mirror the information provided during registration. When you initially submit the paper, suppress author names, affiliations and addresses.
\end{abstract}

\newpage
 
\section{Introdução}

\section{O Problema de Tabela-horário para Universidades (THU)}
\section{Trabalhos Relacionados}
\section{Algoritmos para o THU}
\subsection{Algoritmo Genético}
\subsection{Simulated Annealing}
\subsection{GRASP - com a primeira busca local}
\section{Resultados Computacionais}
\section{Conclusões e Trabalhos Futuros}


O SBPO utilizará este ano, novamente, o sistema JEMS para a submissão de trabalhos. A primeira página dos trabalhos publicados será constituída com as informações fornecidas no formulário de submissão de trabalho do JEMS. Por isto, os nomes de \textbf{todos} os autores devem ser cadastrados nesse formulário. Os nomes  não incluídos nesse formulário não aparecerão entre os autores na programação do Simpósio nem nos Anais.
No campo \textit{Paper Title }deve ser informado apenas o título do trabalho, \textbf{sem qualquer identificação dos autores ou suas instituições}. 
O campo \textit{Paper Abstract} deverá ser preenchido com o Resumo, de \textbf{no máximo 150 palavras}, seguido por 3 palavras-chave e pelo nome da área de classificação principal do trabalho, escolhida entre aquelas assinaladas no campo \textit{Paper Topics}. A seguir, no caso de resumos escritos em português ou espanhol, deverá vir o \textit{Abstract} em inglês e as \textit{keywords}, traduzindo fielmente o Resumo e as palavras-chave. 


\section{Submissão do Texto Completo}

Após cadastrar o artigo, o autor é convidado a carregar para o sistema JEMS arquivo de terminação DOC ou PDF, com o texto completo. A primeira página desse manuscrito deve conter apenas o título do artigo coincidindo exatamente com o informado quando do cadastramento. Pode, também, incluir novamente os resumos e palavras-chave em português ou espanhol e inglês, mas, \textbf{não pode incluir nomes, filiação e endereços de autores}.
Este manuscrito será disponibilizado para o exame pelos revisores, que terão também acesso às informações do cadastro, exceto as referentes aos nomes e instituições dos autores. Uma vez aceito o artigo, os autores serão chamados a encaminhar \textbf{versão final} com a página inicial completa, isto é, com autores, instituições, resumo de no máximo 150 palavras, 3 palavras-chave, \textit{abstract} e \textit{keywords}.
As páginas deste texto não devem vir numeradas, tanto no caso de arquivo enviado quando da submissão quanto no caso do arquivo com a versão final do artigo aceito. A numeração será feita posteriormente para o conjunto de todos os artigos. \textbf{Cabeçalhos e rodapés devem ser deixados em branco}.


\section{Instruções de Formatação}

Os trabalhos completos devem ter, \textbf{no máximo 12 páginas}, tudo incluído neste limite, inclusive tabelas, gráficos, agradecimentos e referências.
Os textos devem utilizar páginas de tamanho \textbf{A4} (29,7 x 21,0 cm) \textbf{com margem superior de 3,3 cm, inferior de 2,5 cm e laterais de 2,9 cm}. Devem ser escritos em coluna única, com fonte \textbf{Times 11}.

\section{Estilo das Citações}

As citações no texto devem conter o último sobrenome do autor, seguido, entre parêntesis, do \textbf{ano da publicação}, como por exemplo \cite{knuth:84}, \cite{boulic:91}, e \cite{smith:99}. As referências no final do texto devem estar em ordem alfabética do último sobrenome do primeiro autor.

\bibliographystyle{sbpo}
\bibliography{sbpo-sample}

\end{document}
