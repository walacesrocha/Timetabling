\documentclass[11pt]{article}

\usepackage{sbpo}
\usepackage{graphicx}
\usepackage{url}
\usepackage{amssymb,amsmath}

\usepackage[T1]{fontenc}
\usepackage[brazil]{babel}   
\usepackage[utf8]{inputenc}
\usepackage[vlined,boxruled,linesnumbered,portuguese]{algorithm2e}



\sloppy

\title{Aplicação das Meta-heurísticas GRASP, Simulated Annealing e Algoritmos Genéticos para o Problema de Tabela-horário para Universidade}

% Uncomment next line to hide authors, affiliations and contacts
%\hideauthors

%{ \author{\sbpoauthor{Walace S. Rocha}
%                   {Núcleo de Processamento de Dados - Universidade Federal do Espírito Santo}
%                   {Vitória -- ES -- Brasil}
%                   {\email{walacesrocha@yahoo.com.br}}
%        \sbpoauthor{Maria C. S. Boeres}
%                   {Departamento de Informática -- Universidade Federal do Espírito Santo}
%                   {Vitória -- ES -- Brasil}
%                   {\email{boeres@inf.ufes.br}}
%        \sbpoauthor{Maria C. Rangel}
%                   {Departamento de Informática -- Universidade Federal do Espírito Santo}
%                   {Vitória -- ES -- Brasil}
%                   {\email{crangel@inf.ufes.br}}
%        \sbpoauthor{Leandro B. Ferreira}
%                   {Curso de Ciência da Computação -- Universidade Federal do Espírito Santo}
%                   {Vitória -- ES -- Brasil}
%                   {\email{leandrobfe@gmail.com}}}

\begin{document} 

\maketitle

\palavrachave{Tabela-horário de Universidade}
\palavrachave{Algoritmos Genético e \textit{Simulated Annealing}}
\palavrachave{GRASP}
\areaprincipal{Otimização Combinatória}

\begin{resumo} 
%  Este modelo resume as normas de formatação para os trabalhos completos a serem publicados nos Anais do XLI SBPO. O Resumo deve ter no máximo 150 palavras. Título, nomes dos autores, sua filiação e endereços, resumo e palavras-chave devem repetir fielmente o que foi informado quando o artigo foi cadastrado através do sistema JEMS. No primeiro upload, por ocasião da submissão do trabalho, exclua desta primeira página os nomes, instituições e endereços dos autores.

 Este trabalho apresenta algumas meta-heurísticas usadas para resolver o problema de tabela-horário para universidades. Um Algoritmo Genético e um \textit{Simulated Annealing}, apresentados na literatura, são modificados para produzirem melhores soluções. Um algoritmo GRASP com \textit{Path-Relinking} é proposto para o problema. A formulação do problema utilizada é a adotada no ITC-2007. Resultados computacionais das três implementações são comparados entre si e com as melhores respostas encontradas na literatura, para o conjunto de instâncias do ITC-2007. Algumas propostas de melhoria no GRASP são apresentadas como metas futuras.
\end{resumo}

\keyword{University Timetabling}
\keyword{Genetic and Simulated Annealing Algorithms}
\keyword{GRASP}

\mainarea{Combinatorial Optimization}

\begin{abstract}
  This work presents some metaheuristics used to solve the university timetabling problem. A Genetic and a Simulated Annealing algorithms, presented in literature, are modified in order to produce better solutions. A GRASP with Path-Relinking is proposed for this problem. The problem formulation is the same adopted in ITC-2007. Computational results of the three implementations are compared with each other and with the best solutions found in the literature  for the ITC-2007 instances set. Some improvement proposals for GRASP are presented as future goals.
%  This model summarizes the formatting rules for full papers accepted for inclusion ion the Proceedings of the XLI SBPO. The Abstract must have no more than 150 words Title, author names, affiliations, addresses, abstract and keywords must precisely mirror the information provided during registration. When you initially submit the paper, suppress author names, affiliations and addresses.
\end{abstract}

\newpage
 
\section{Introdução}

Um dos problemas de grande interesse na área de otimização combinatória é o problema de tabela-horário \cite{Schaerf95asurvey}, \cite{Lewis2007:survey}. Dado um conjunto de disciplinas, alunos, professores e espaço físico, o problema consiste basicamente em alocar os horários das disciplinas levando-se em conta o espaço e horários disponíveis, além de respeitar um conjunto de restrições fortes e fracas que surgem das particularidades de cada instituição. As restrições fortes são aquelas que sempre devem ser satisfeitas, como por exemplo, exigir que um aluno não possa assistir mais de uma aula no mesmo horário. Uma tabela-horário é dita viável quando não viola nenhuma restrição forte. As fracas são restrições que não geram inviabilidade, mas refletem algumas preferências dos professores, alunos ou até mesmo da escola, por exemplo, a penalização de uma tabela-horário que tenha aulas isoladas. Quanto mais restrições fracas forem satisfeitas, melhor será a tabela-horário. A formulação usada neste trabalho é a adotada pela competição ITC-2007 \cite{itc2007}. É importante ressaltar que o ITC-2007 usou três formulações independentes de problema de tabela-horário. A primeira formulação é voltada para tabela-horário de exames. A segunda e a terceira são voltadas para a tabela-horário semestral de universidades. A diferença é que a segunda tem foco nas informações de matrícula de alunos enquanto a terceira é baseada nos currículos da universidade. Esse trabalho é direcionado à resolução da terceira formulação.

A meta-heurística GRASP (\textit{Greedy Randomized Adaptive Search Procedure}) \cite{grasp_resende_ribeiro} é uma técnica de destaque no ramo de otimização combinatória. Ela tem sido aplicada em problemas de cobertura de conjuntos \cite{Res98}, satisfabilidade \cite{FesParPitRes06a}, árvore geradora mínima \cite{Souza02agrasp}, entre outros. Contudo, há pouca pesquisa desse algoritmo nos problemas de tabela-horário. Foram encontradas apenas implementações para a modelagem de escolas \cite{Souza:2004} e \cite{Vieira_agrasp}.

Este trabalho implementa um algoritmo GRASP para geração de tabela-horário usando a terceira formulação do ITC-2007. O algoritmo possui uma fase inicial onde é garantido que uma tabela-horário viável será encontrada. Na segunda fase ocorre o melhoramento da tabela-horário através de busca local. Uma terceira fase de intensificação ocorre usando o procedimento \textit{Path-Relinking}. Esse procedimento de três fases se repete algumas iterações, sempre gerando uma nova tabela. A melhor de todas encontradas é a resposta final. Além do GRASP, são implementados um algoritmo genético \cite{massoodian2008} e um \textit{Simulated Annealing} \cite{CDGS11b}, com proposta de melhorias. Os resultados obtidos são comparados com as melhores soluções publicadas na literatura para este problema.

Na seção \ref{sec:problema} é apresentada a formulação do problema adotada neste trabalho. Na seção \ref{sec:trabalhos_relacionados} são listadas algumas técnicas que tem sido usadas para resolver o problema de tabela-horário de universidades, tanto na formulação do ITC-2007 como em outras formulações. Na seção \ref{sec:algoritmos} são apresentados três algoritmos implementados para o problema: algoritmo genético, \textit{simulated annealing} e o GRASP. Na seção \ref{sec:resultados} são apresentados os resultados obtidos com as diferentes técnicas. Os resultados são confrontados com as melhores respostas para o problema encontradas na literatura. Por fim, a seção \ref{sec:conclusao} apresenta as conclusões obtidas com o trabalho e lista algumas melhorias futuras para o algoritmo GRASP.

\section{O Problema de Tabela-horário para Universidades (THU)}
\label{sec:problema}

Apesar das formulações encontradas na literatura para o problema de tabela-horário serem bastante variadas, elas podem ser agrupadas em três grupos: tabela-horário de escolas, tabela-horário de universidades e tabela-horário para aplicação de exames. Neste trabalho é desenvolvido um algoritmo para tratar a alocação de tabela-horário de universidades (THU). A formulação do problema é a mesma adotada na competição ITC-2007. A principal vantagem em se adotar esta formulação é que muitos autores a utilizam, contribuindo para comparação de resultados com diversos trabalhos de grande relevância na literatura.

O problema de tabela-horário na formulação ITC-2007 utiliza os seguintes parâmetros:

\begin{itemize}

\item Dias, Horários e Períodos: É dado o número de dias na semana em que há aula (geralmente cinco ou seis). Um número fixo de horários de aula, igual para todos os dias, é estabelecido. Um período é um par composto de um dia e um horário. O total de períodos é obtido multiplicando a quantidade de dias pela quantidade de horários do dia.

\item Disciplinas e Professores: Cada disciplina possui uma quantidade de aulas semanais que devem ser alocadas em períodos diferentes. É lecionada por um professor e assistida por um dado número de alunos. Um número mínimo de dias é determinado para a distribuição de suas aulas na semana, além de ser possível a existência de períodos indisponíveis para sua atribuição.

\item Salas: Cada sala possui uma capacidade diferente de assentos.

\item Currículo: Um currículo é um grupo de disciplinas que possuem alunos em comum.

\end{itemize}

Uma solução do THU consiste na alocação das aulas das disciplinas a um conjunto de períodos e uma sala. As restrições do problema, fortes e fracas, são descritas a seguir:

\subsection{Restrições Fortes (RFt)}

\begin{itemize}

\item Aulas: Todas as aulas das disciplinas devem ser alocadas e em períodos diferentes. Uma violação ocorre se uma aula não é alocada ou se duas aulas distintas são alocadas no mesmo período. (RFt1)

\item Conflitos: Aulas de disciplinas do mesmo currículo ou lecionadas pelo mesmo professor devem ser alocadas em períodos diferentes. (RFt2)

\item Ocupação de Sala: Duas aulas não podem acontecer na mesma sala e no mesmo período. (RFt3)

\item Disponibilidade de Professor: Se um professor é indisponível em um horário, nenhuma disciplina que ele leciona deve ser alocada naquele horário. (RFt4)

\end{itemize}

\subsection{Restrições Fracas (RFc)}

\begin{itemize}

\item Dias Mínimos de Trabalho: As aulas de cada disciplina devem ser espalhadas por uma quantidade mínima de dias. Cada dia abaixo do mínimo é contado como uma violação. (RFc1)

\item Aulas Isoladas: Aulas do mesmo currículo devem ser alocadas em períodos adjacentes. Cada aula isolada é contada como uma violação. (RFc2)

\item Capacidade da Sala: O número de alunos da disciplina deve ser menor ou igual ao número de assentos da sala em que a aula for dada. Cada aluno excedente contabiliza uma violação. (RFc3)

\item Estabilidade de Sala: Todas as aulas de uma disciplina devem ser dadas na mesma sala. Cada sala distinta é contada como uma violação. (RFc4)

\end{itemize}

Na contagem total das violações fracas são considerados pesos diferentes para cada tipo de violação. A restrição de dias mínimos possui peso cinco, aulas isoladas, peso dois e as demais, peso um.

Uma solução viável do THU deve atender a todas as restrições fortes. Uma solução ótima do THU é viável e minimiza a função objetivo apresentada na equação \ref{funcaoObjetivo}:

\begin{equation}\label{funcaoObjetivo}
%\begin{split}
f = \text{Violações}_{RFt} + \text{Violações}_{RFc}
%\end{split}
\end{equation}

\noindent onde $\text{Violações}_{RFt} = |RFt1|_v + |RFt2|_v + |RFt3|_v + |RFt4|_v$, $\text{Violações}_{RFc} = 5|RFc1|_v + 2|RFc2|_v + |RFc3|_v + |RFc4|_v$ e $|.|_v$ representa o número de violações.

%\section{Formulação Matemática}
%\label{sec:formulacao_matematica}

%A formulação adotada para o THU é proposta em \cite{ref}, que será reproduzida a seguir:

\section{Trabalhos Relacionados}
\label{sec:trabalhos_relacionados}

As principais técnicas de solução do problema THU recaem basicamente em três grupos: programação matemática, programação em lógica e meta-heurísticas. Por ser um problema complexo, é inviável computacionalmente buscar a solução ótima. Alguns métodos tem alcançado este objetivo, mas apenas para instâncias pequenas. Esse problema é classificado como NP-completo \cite{Schaerf95asurvey}.

Na área de programação matemática, bons resultados têm sido obtidos com programação inteira. Em \cite{lach_lubbecke} pode ser vista uma formulação completa do problema. Ela foi executada com CPLEX9 e conseguiu soluções com respostas bem próximas às melhores obtidas na competição ITC-2007. Em \cite{Burke_abranch-andcut} pode ser vista outra formulação com algumas relaxações. Com um tempo de execução aproximado de quinze minutos no CPLEX10, esse algoritmo conseguiu encontrar a solução ótima para duas instâncias da competição, além de encontrar limites inferiores para as outras instâncias.

Alguns resultados relevantes também tem sido encontrados com programação em lógica. \cite{springerlink:10.1007/s10479-012-1081-x} apresenta uma formulação usando \textit{MaxSAT} em que se conseguiu empatar ou melhorar as respostas de quase metade das instâncias do ITC-2007. \cite{Gueret95buildinguniversity} e \cite{Goltz99universitytimetabling} adotam formulações diferentes do ITC-2007, mas destacam como programação em lógica combinada com programa por restrições pode implementar modelos bem flexíveis, em que restrições podem ser adicionadas, modificadas ou excluídas com pouca alteração de código-fonte.

A maioria das técnicas de solução deste problema utilizam meta-heurísticas. Ainda não foi identificada uma meta-heurística que seja considerada a melhor.  Algoritmos eficientes que produzem bons resultados têm sido encontrados usando \textit{Simulated Annealing} \cite{3-phaseSA}, \cite{sa_hyper_heuristica}, \cite{Elmohamed98acomparison}, Algoritmo Genético \cite{Erben95agenetic}, \cite{suyanto}, \cite{Kanoh:2008:KGA:1460198.1460201}, Busca Tabu \cite{elloumi2008}, entre outras mais recentes como a Busca de Harmonia \cite{albetar_harmony}.

Há também algumas técnicas híbridas em que são usadas mais de uma meta-heurística ou aplicadas de forma diferente. Exemplos deste tipo de implementação podem ser visto em \cite{massoodian2008} em que o algoritmo genético é combinado com um algoritmo de busca local para melhorar a qualidade das soluções. Em \cite{3-phaseSA} há um algoritmo que constrói a tabela-horário em três etapas, cada uma executando um procedimento de \textit{Simulated Annealing} independente.

Este trabalho propõe um algoritmo GRASP para o problema de tabela-horário de universidades. Há na literatura algumas propostas do GRASP, mas para tabela-horário de escolas \cite{Souza:2004} e \cite{Vieira_agrasp}. Além disso propõe melhorias para um algoritmo genético e um \textit{Simulated Annealing} encontrados na literatura e aplicados à mesma formulação do problema.


\section{Algoritmos para o THU}
\label{sec:algoritmos}

Como foi apresentado na seção anterior, muitas meta-heurísticas têm sido usadas para resolver o problema de THU. Dentre elas, Algoritmo Genético e \textit{Simulated Annealing} têm se destacado pela quantidade de propostas que implementam essas técnicas. Dentre estas propostas, foram escolhidas duas consideradas promissoras, com o intuito de estudá-las mais profundamente e implementar melhorias para produzir soluções melhores.

Os algoritmos escolhidos e as melhorias implementadas são apresentados nas próximas subseções. Por fim, um novo algoritmo para o problema de THU é apresentado usando a meta-heurística GRASP.

\subsection{Algoritmo Genético (AG)}

O algoritmo genético escolhido para estudo é o proposto em \cite{massoodian2008}. Após a geração de uma população inicial, são realizadas as iterações (gerações) onde são aplicados os procedimentos de seleção, cruzamento e mutação. Ao final de cada geração, uma busca local é feita no melhor indivíduo da população.

A primeira melhoria foi na geração da população inicial. Na versão original, somente uma restrição forte é considerada nesta etapa. No algoritmo modificado procura-se respeitar todas as restrições fortes. Devido à complexidade das instâncias nem sempre é possível respeitar todas elas, mas esta modificação produziu soluções mais próximas da viabilidade do que o original.

Uma falha identificada no procedimento de cruzamento original é que, em geral, ele produz filhos com muitas violações das restrições fortes. Para sanar este problema foi implementado um procedimento de reparação do filho gerado. Ele verifica as aulas que estão em horários inviáveis e procura um novo horário em que ela pode ser inserida.

O procedimento de mutação também foi modificado para não permitir uma alteração no indivíduo que produza novas inviabilidades. A modificação proposta fez com que o algoritmo genético encontrasse uma solução viável com menos iterações.

\subsection{\textit{Simulated Annealing} (SA)}

O algoritmo \textit{Simulated Annealing} escolhido para estudo foi o proposto por \cite{CDGS11b}. Apesar do algoritmo proposto resolver a segunda formulação do ITC-2007, suas idéias principais foram aproveitadas para resolver a terceira formulação, que é o objetivo deste trabalho.

O algoritmo original segue a idéia básica do \textit{Simulated Annealing}. Uma solução inicial é gerada procurando violar o menos possível as restrições fortes. Em seguida começa o processo de redução da temperatura em que a solução inicial vai sendo melhorada até chegar a uma temperatura final. Os valores das temperaturas inicial e final são definidos na entrada. A temperatura é reduzida a cada iteração de acordo com a seguinte relação: $T_{i+1} = \beta \times T_i, 0 < \beta < 1$ (resfriamento geométrico).

A mudança implementada é referente ao cálculo da vizinhança da solução atual. No algoritmo original o vizinho é calculado com um ou dois passos, chamados de \textit{MOVE} e \textit{SWAP}. O primeiro passo sempre é realizado, enquanto o segundo só ocorre de acordo uma certa probabilidade. O primeiro passo move uma aula para uma posição vazia. O segundo passo troca duas aulas de posição na tabela. No algoritmo modificado só ocorre um passo para determinar o vizinho. Pode ser um \textit{MOVE} ou um \textit{SWAP}, com igual probabilidade.

Testes computacionais mostraram que esta pequena modificação produziu resultados cerca de $60\%$ melhores que o algoritmo original.

\subsection{\textit{Greedy Randomized Adaptive Search Procedure} (GRASP)}

Cada iteração GRASP \cite{grasp_resende_ribeiro} possui basicamente duas etapas: na primeira uma solução inicial é construída enquanto na segunda a solução é melhorada através de um algoritmo de busca local. Um número máximo de iterações é estabelecido. A melhor solução encontrada em uma destas iterações é a solução final.

\subsubsection{Geração da Solução Inicial}

A geração da solução inicial do GRASP é um algoritmo construtivo baseado no princípio guloso aleatório. Partindo de uma tabela-horário vazia, as aulas são acrescentadas uma a uma até que todas estejam alocadas. A escolha é tanto gulosa (para produzir soluções de boa qualidade) quanto aleatória (para produzir soluções diversificadas).

O principal objetivo deste procedimento é produzir uma solução viável, ou seja, sem violações fortes. Com intuito de alcançar este objetivo, é adotada uma estratégia de alocar as aulas mas conflitantes inicialmente. Poucos horários são viáveis para as disciplinas mais conflitantes, portanto, é melhor alocá-las quando a tabela está mais vazia.

As aulas que ainda estão desalocadas são mantidas numa lista organizada por ordem de dificuldade. Quanto menos horários disponíveis existem para a aula e mais presente ela está em currículos diferentes, mais difícil é encontrar um horário viável.

Em cada iteração, uma aula desalocada é inserida na tabela. A aula mais difícil é escolhida para ser alocada. Existem diferentes combinações de horários e salas para a alocação. Os custos de todas essas combinações são calculados levando-se em conta as penalizações das restrições fracas. As combinações que possuem horários inviáveis são descartadas. Com base no menor e maior custo de adição de um elemento à solução ($c^{min}$ e $c^{max}$) é construída a lista restrita de candidatos (LRC). Estarão na LRC as aulas cujos custos estejam no intervalo \begin{math} [c^{min}, c^{min}+\alpha(c^{max} - c^{min})]\end{math}. Com $\alpha=0$, o algoritmo é puramente guloso, enquanto com $\alpha=1$ a construção é aleatória. Uma aula é escolhida aleatoriamente da LRC e acrescentada à solução. Em alguns casos, no entanto, não há uma posição viável na solução para inserção. Para contornar esta situação foi implementado um procedimento denominado “explosão”. É uma estratégia que retira da tabela uma aula alocada anteriormente para abrir espaço para a aula que não está sendo possível alocar. Essa estratégia mostrou-se bastante eficaz, tornando possível gerar uma tabela viável para todas as instâncias testadas.

\subsubsection{Busca Local}

As soluções geradas pela fase inicial são viáveis, porém não são necessariamente ótimas. Uma busca local é usada para explorar a vizinhança e encontrar soluções melhores. O fator determinante nesta etapa é como a vizinhança será explorada, ou que movimentos serão aplicados para se encontrar um vizinho.

Neste trabalho um vizinho é gerado aplicando dois movimentos sucessivos:

\begin{itemize}
\item \textit{\textbf{MOVE}}: Uma aula é movida para uma posição vazia da tabela.
\item \textit{\textbf{SWAP}}: Duas aulas são trocadas de posição na tabela.
\end{itemize}

O segundo movimento (\textit{SWAP}) é aplicado com uma certa probabilidade, portanto alguns vizinhos são gerados apenas com \textit{MOVE}, outros com \textit{MOVE} seguido de \textit{SWAP}. A taxa de \textit{SWAP} é parametrizada. As escolhas de aulas e posições vazias para aplicação dos movimentos são todas aleatórias.

A busca local parte da solução inicial e em cada iteração gera um vizinho da solução atual. É contado o número de iterações sem melhora. Sempre que um vizinho com melhor valor de função objetivo é encontrado, é atualizado como solução atual e a contagem de iterações sem melhora é zerada. O teste de parada da busca local consiste no número de iterações sem melhora, que é um parâmetro do algoritmo. 

\subsection{\textit{Path-Relinking}}

A proposta original do GRASP é considerada sem memória, pelo fato das soluções obtidas nas iterações não serem compartilhadas entre si para produzir melhores soluções. O algoritmo \textit{Path-Relinking} \cite{Glover96tabusearch}, \cite{pathRelinking} trabalha com duas soluções: uma ótima local e outra elite, tentando melhorar a solução ótima local usando a estrutura da solução elite. Para aplicar o \textit{Path-Relinking}, o GRASP mantém um conjunto de soluções elite. A cardinalidade deste conjunto é limitada e contém as soluções que foram encontradas com menor valor de função objetivo.

Este procedimento constrói um caminho conectando uma solução inicial e uma solução alvo. Esse caminho é feito aplicando movimentos que façam com que a solução inicial fique igual a solução alvo. Durante o percurso melhores soluções podem ser encontradas. Caminhando no sentido direto, a solução inicial é a ótima local e a alvo uma solução elite. No sentido inverso, o caminho é percorrido da solução elite para a ótima local. A solução elite que será usada no procedimento é escolhida aleatoriamente.

O algoritmo \ref{algoGrasp} apresenta o pseudo-código do algoritmo GRASP implementado para o THU. São dados como entrada a instância do problema, o número máximo de iterações e o tamanho do conjunto de soluções elite. Nas linhas 1 e 2, a função $f^*$ e o conjunto \textit{Pool} são inicializados. Na linha 3 começa o ciclo de iterações GRASP. Em cada iteração ocorre a geração de uma solução inicial (linha 4) e a busca local (linha 5). O \textit{Path-Relinking} só começa a ser executado a partir do momento em que o \textit{Pool} de soluções elites está completo (linha 6), e isto ocorre a partir da iteração número $TamPool$, pois em cada iteração uma nova solução é inserida no \textit{Pool}. Na linha 7 é escolhida aleatoriamente uma solução elite que será usado no procedimento. A outra solução é a ótima local obtida na busca local (linha 5). Na linha 10 verifica se a solução encontrada é a melhor até o momento. Se for, esta solução é armazenada como a melhor. No final da iteração (linha 14) o \textit{Pool} é atualizado. Nas primeiras iterações toda nova solução é inserida no \textit{Pool}. Quando ele já está completo, a nova solução encontrada só entra se ela for melhor que alguma que estiver lá. Neste caso a pior solução do \textit{Pool} é retirada.

\begin{algorithm}[H]
\label{algoGrasp}
\SetAlgoLined
\SetKwFunction{GeraSolucaoInicial}{GeraSolucaoInicial}
\SetKwFunction{BuscaLocal}{BuscaLocal}
\SetKwFunction{PathRelinking}{PathRelinking}
\SetKwFunction{AtualizaPool}{AtualizaPool}
\Entrada{Instância p, MaxIter, TamPool}
\Saida{Solução $S^{*}$}

$f^{*} \leftarrow \infty $ \;
$Pool \leftarrow \phi $ \;
\Para{$i\leftarrow 1$ \Ate $MaxIter$}{
	$S \leftarrow GeraSolucaoInicial() $ \;
	$S \leftarrow BuscaLocal(S) $ \;
	\Se{$ i > TamPool $}{
		Selecione aleatoriamente $S_{elite} \in Pool $ \;
		$S \leftarrow PathRelinking(S, S_{elite}) $ \;
	}
	\Se{$ f(S) < f^{*} $}{
		$S^{*} \leftarrow S $ \;
		$f^{*} \leftarrow f(S) $ \;
	}
	$AtualizaPool(S, Pool)$ \;
}
\caption{Algoritmo GRASP}
\end{algorithm}


\section{Resultados Computacionais}
\label{sec:resultados}

Os algoritmos modificados AG e SA e o GRASP proposto foram testados com as mesmas 21 instâncias que foram usadas na competição ITC-2007. Após a realização da competição foi criado um \textit{web site} para reunir algumas informações relacionadas ao problema. Neste \textit{site} \cite{ctt}, podem ser obtidas as instâncias usadas na competição, verificar limites inferiores para cada uma delas, gerar novas instâncias além de verificar as melhores soluções encontradas por diferentes pesquisadores para cada instância.

Todos os algoritmos descritos neste trabalho foram implementados na linguagem C, compilados com GCC 4.1.2 e testados em máquina Linux com a distribuição Fedora Core 8, com processador Intel quad-core 2.4 GHz e 2 Gb de memória RAM.

Não foi levado em consideração nos testes o tempo de execução dos algoritmos, pois no \textit{site} \cite{ctt} só é comparado a qualidade das respostas.

Testes preliminares para calibragem dos parâmetros do GRASP mostraram que quanto menor o valor de $\alpha$ na montagem da lista restrita dos candidatos (LRC), melhor é a qualidade da solução inicial. Contudo, mesmo com um $\alpha$ pequeno a solução inicial ainda é longe da ótima. Sendo assim, optou-se por um valor de $\alpha = 0.15$. Este valor não é muito alto para gerar soluções iniciais boas, e também não é tão próximo de zero para gerar soluções diversificadas. Na busca local foi definido número máximo de iterações sem melhora igual a 10000. A geração dos vizinhos é feita de duas formas diferentes. Uma utiliza somente o \textit{MOVE}, enquanto que a outra usa o \textit{MOVE} seguido de um \textit{SWAP}. As duas formas tem igual probabilidade de serem escolhidas para geração do vizinho. Com relação ao procedimento \textit{Path-Relinking}, testes mostraram que o sentido inverso produzem melhores resultados para o problema. \cite{grasp_resende_ribeiro} comentam que, em geral, boas soluções estão mais próximas às soluções elite do que às soluções ótimas locais. O \textit{Pool} de soluções elite foi configurado com tamanho 5.

A tabela \ref{tabResultados} lista as melhores soluções encontradas por cada algoritmo implementado. A coluna CTT se refere a melhor solução encontrada na internet \cite{ctt}.

\begin{center}
\begin{table}[htbf]
\begin{tabular}{|l|c|c|c|c|}
\hline
Instância & CTT & AG & SA & GRASP \\ \hline
comp01 & 5 & 15 & \textbf{6} & \textbf{6} \\
comp02 & 24 & 234 & \textbf{116} & 131 \\
comp03 & 66 & 233 & \textbf{116} & 141 \\
comp04 & 35 & 137 & \textbf{76} & 78 \\
comp05 & 290 & 818 & \textbf{429} & 552 \\
comp06 & 27 & 215 & 132 & \textbf{123} \\
comp07 & 6 & 205 & \textbf{99} & 120 \\
comp08 & 37 & 169 & \textbf{84} & 87 \\
comp09 & 96 & 229 & \textbf{137} & 164 \\
comp10 & 4 & 183 & \textbf{67} & 78 \\
comp11 & 0 & 13 & \textbf{0} & \textbf{0} \\ \hline
\end{tabular}
\medskip
\begin{tabular}{|l|c|c|c|c|}
\hline
Instância & CTT & AG & SA & GRASP \\ \hline
comp12 & 300 & 544 & \textbf{407} & 517 \\
comp13 & 59 & 206 & \textbf{106} & 120 \\
comp14 & 51 & 173 & \textbf{90} & 98 \\
comp15 & 66 & 225 & \textbf{120} & 138 \\
comp16 & 18 & 198 & \textbf{91} & 104 \\
comp17 & 56 & 206 & \textbf{122} & 156 \\
comp18 & 62 & 153 & 133 & \textbf{115} \\
comp19 & 57 & 207 & \textbf{111} & 139 \\
comp20 & 4 & 257 & \textbf{130} & 149 \\
comp21 & 76 & 276 & \textbf{151} & 188 \\
 &  &  &  &   \\ \hline
\end{tabular}
\caption{Resultados computacionais - Valor da função objetivo obtida pelo algoritmo genético (AG), \textit{simulated annealing} (SA), GRASP e as melhores soluções disponíveis no site CTT}
\label{tabResultados}
\end{table}
\end{center}


%\begin{table}

%\centering
%\begin{tabular}{|l|c|c|c|c|c|c|} 
%\hline

%Instância & AG & SA & GRASP & CTT &  \\ \hline

%comp01 & 15 & 6 & 6 & 5 & \\
%comp02 & 234 & 116 & 131 & 24 & \\
%comp03 & 233 & 116 & 141 & 66 & \\
%comp04 & 137 & 76 & 78 & 35 & \\
%comp05 & 818 & 429 & 552 & 290 & \\
%comp06 & 215 & 132 & 123 & 27 & \\
%comp07 & 205 & 99 & 120 & 6 & \\
%comp08 & 169 & 84 & 87 & 37 & \\
%comp09 & 229 & 137 & 164 & 96 & \\
%comp10 & 183 & 67 & 78 & 4 & \\
%comp11 & 13 & 0 & 0 & 0 & \\
%comp12 & 544 & 407 & 517 & 300 & \\
%comp13 & 206 & 106 & 120 & 59 & \\
%comp14 & 173 & 90 & 98 & 51 & \\
%comp15 & 225 & 120 & 138 & 66 & \\
%comp16 & 198 & 91 & 104 & 18 & \\
%comp17 & 206 & 122 & 156 & 56 & \\
%comp18 & 153 & 133 & 115 & 62 & \\
%comp19 & 207 & 111 & 139 & 57 & \\
%comp20 & 257 & 130 & 149 & 4 & \\
%comp21 & 276 & 151 & 188 & 76 & \\
%\hline
%\end{tabular}
%\caption{Resultados computacionais - Quantidade de penalizações obtidas pelo algoritmo genético (AG), \textit{simulated annealing} (SA), GRASP e as melhores soluções disponíveis no site CTT}
%\label{tabResultados}
%\end{table} 

%\begin{table}
%\tiny
%\centering
%\begin{tabular}{|l|c|c|c|c|c|c|c|c|c|c|c|c|c|c|c|c|c|c|c|c|c|} 
%\hline
% comp01 & comp02	& comp03 & comp04 & comp05 & comp06	& comp07 & comp08 & comp09 & comp10	& comp11 & comp12 & comp13 & comp14	& comp15 & comp16 & comp17 & comp18	& comp19 & comp20 & comp21 \\
% & 01 & 02	& 03 & 04 & 05 & 06	& 07 & 08 & 09 & 10	& 11 & 12 & 13 & 14	& 15 & 16 & 17 & 18	& 19 & 20 & 21 \\ \hline
%CTT & 5 & 24 & 66 & 35 & 290 & 27 & 6 & 37 & 96 & 4 & 0 & 300 & 59 & 51 & 66 & 18 & 56 & 62 & 57 & 4 & 76 \\ \hline
%AG & 15	& 234 & 233 & 137 & 818 & 215 & 205 & 169 & 229 & 183 & 13 & 544 & 206 & 173 & 225 & 198 & 206 & 153 & 207 & 257 & 276 \\
%SA & 6* &	116* & 116* & 76* & 429* & 132 & 99* & 84* & 137* & 67* & 0* & 407* & 106* & 90* & 120* & 91* & 122* & 133 & 111* & 130* & 151* \\
%GRASP & 6* & 131 & 141 & 78 & 552 & 123* & 120 & 87 & 164 & 78 & 0* & 516 & 120 & 98 & 138 & 104 & 156 & 115* &	139 & 141 & 188 \\
%\hline
%\end{tabular}
%\caption{Resultados computacionais - Quantidade de penalizações obtidas pelo algoritmo genético (AG), \textit{simulated annealing} (SA), GRASP e as melhores soluções disponíveis no site CTT}
%\label{tabResultados}
%\end{table} 


\section{Conclusões e Trabalhos Futuros}
\label{sec:conclusao}

Os resultados da tabela \ref{tabResultados} mostraram que \textit{Simulated Annealing} e GRASP obtiveram os melhores resultados, sendo o \textit{Simulated Annealing} superior em 17 instâncias, o GRASP superior em 2 instâncias e empate em outras 2 instâncias. O algoritmo genético obteve os piores resultados em todas as instâncias.

Pode-se notar que para muitas instâncias a melhor resposta encontrada pelos algoritmos ainda está longe das melhores respostas conhecidas. Visando a melhoria dos métodos implementados será necessário investigar formas diferentes de explorar a vizinhança. Observamos, nas vizinhanças de todos os algoritmos implementados que em geral é necessário gerar muitos vizinhos para encontrar melhora, principalmente nas instâncias mais complexas que possuem muitas disciplinas.

Uma vizinhança similar à que foi implementada no \textit{Simulated Annealing} está sendo implementada no GRASP. O vizinho não seria gerado com duas etapas, mas apenas uma: ou com \textit{MOVE} ou com \textit{SWAP}, com igual probabilidade.

Outra proposta de melhoria no GRASP é a paralelização do algoritmo. A estrutura do algoritmo permite uma paralelização trivial em que cada nó de processamento executa uma iteração independente. No final da execução os melhores resultados são comparados. Em \cite{grasp_paralelo} podem ser encontradas algumas estratégias mais complexas de paralelização do algoritmo.


%O SBPO utilizará este ano, novamente, o sistema JEMS para a submissão de trabalhos. A primeira página dos trabalhos publicados será constituída com as informações fornecidas no formulário de submissão de trabalho do JEMS. Por isto, os nomes de \textbf{todos} os autores devem ser cadastrados nesse formulário. Os nomes  não incluídos nesse formulário não aparecerão entre os autores na programação do Simpósio nem nos Anais.
%No campo \textit{Paper Title }deve ser informado apenas o título do trabalho, \textbf{sem qualquer identificação dos autores ou suas instituições}. 
%O campo \textit{Paper Abstract} deverá ser preenchido com o Resumo, de \textbf{no máximo 150 palavras}, seguido por 3 palavras-chave e pelo nome da área de classificação principal do trabalho, escolhida entre aquelas assinaladas no campo \textit{Paper Topics}. A seguir, no caso de resumos escritos em português ou espanhol, deverá vir o \textit{Abstract} em inglês e as \textit{keywords}, traduzindo fielmente o Resumo e as palavras-chave. 


%\  \section{Submissão do Texto Completo}

%Após cadastrar o artigo, o autor é convidado a carregar para o sistema JEMS arquivo de terminação DOC ou PDF, com o texto completo. A primeira página desse manuscrito deve conter apenas o título do artigo coincidindo exatamente com o informado quando do cadastramento. Pode, também, incluir novamente os resumos e palavras-chave em português ou espanhol e inglês, mas, \textbf{não pode incluir nomes, filiação e endereços de autores}.
%Este manuscrito será disponibilizado para o exame pelos revisores, que terão também acesso às informações do cadastro, exceto as referentes aos nomes e instituições dos autores. Uma vez aceito o artigo, os autores serão chamados a encaminhar \textbf{versão final} com a página inicial completa, isto é, com autores, instituições, resumo de no máximo 150 palavras, 3 palavras-chave, \textit{abstract} e \textit{keywords}.
%As páginas deste texto não devem vir numeradas, tanto no caso de arquivo enviado quando da submissão quanto no caso do arquivo com a versão final do artigo aceito. A numeração será feita posteriormente para o conjunto de todos os artigos. \textbf{Cabeçalhos e rodapés devem ser deixados em branco}.


%\section{Instruções de Formatação}

%Os trabalhos completos devem ter, \textbf{no máximo 12 páginas}, tudo incluído neste limite, inclusive tabelas, gráficos, agradecimentos e referências.
%Os textos devem utilizar páginas de tamanho \textbf{A4} (29,7 x 21,0 cm) \textbf{com margem superior de 3,3 cm, inferior de 2,5 cm e laterais de 2,9 cm}. Devem ser escritos em coluna única, com fonte \textbf{Times 11}.

%\section{Estilo das Citações}

%As citações no texto devem conter o último sobrenome do autor, seguido, entre parêntesis, do \textbf{ano da publicação}, como por exemplo \cite{knuth:84}, \cite{boulic:91}, e \cite{smith:99}. As referências no final do texto devem estar em ordem alfabética do último sobrenome do primeiro autor.

\bibliographystyle{sbpo}
\bibliography{sbpo-sample}

\end{document}
